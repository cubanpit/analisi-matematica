\begin{figure}
\tikzsetnextfilename{radici-complesse}
\centering
\begin{tikzpicture}
\draw (0,0) circle (2cm);
\draw (0,0) node[above left]{$O$};
\draw [->] (-2.5cm,0cm) -- (2.5cm,0cm) node[right]{$\mathrm{Re}$};
\draw [->] (0cm,-2.5cm) -- (0cm,2.5cm) node[above]{$\mathrm{Im}$};
\draw [dashed] (60:2cm) -- (180:2cm) -- (300:2cm) -- (60:2cm);
\draw (60:2cm) node[small dot,pin={[pin distance=.3cm]60:$e^{i\frac{\pi}3}$}]{};
\draw (180:2cm) node[small dot,pin={[pin distance=.3cm]135:$-1$}]{};
\draw (300:2cm) node[small dot,pin={[pin distance=.3cm]300:$e^{-i\frac{\pi}3}$}]{};
\end{tikzpicture}
\caption{Le tre radici cubiche di -1, che formano un triangolo equilatero inscritto nella circonferenza di raggio $r=1$.}
\end{figure}