\chapter{Forme differenziali lineari}
\section{Insiemi connessi}
Prima di introdurre la teoria delle forme differenziali, introduciamo la definizione di insieme \emph{connesso}.
\begin{definizione} \label{d:connesso}
	Un insieme $A\subset\R^n$ aperto si dice connesso se laddove può essere scritto come unione di due insiemi aperti e disgiunti, uno dei due è vuoto, vale a dire
	\begin{equation*}
		A=A_1\cup A_2\text{ e }A_1\cap A_2=\emptyset\text{, allora }A_1=\emptyset\text{ o }A_2=\emptyset.
	\end{equation*}
\end{definizione}
Un'altra definizione si ottiene se consideriamo gli archi che collegano due punti dell'insieme: se per qualsiasi coppia di punti l'arco è contenuto tutto nell'insieme, anche in questo caso l'insieme si dice connesso
Il seguente teorema mostra che queste due definizioni sono in realtà interscambiabili.
\begin{teorema} \label{t:connessione-per-archi}
	Sia $A$ un insieme aperto e connesso di $\R^n$.
	Per qualsiasi coppia di punti $\vec p_1,\vec p_2\in A$ esiste un arco poligonale con estremi $\vec p_1$ e $\vec p_2$ tutto contenuto in $A$.
\end{teorema}
\begin{proof}
	Sia $\vec p_1\in A$, e chiamiamo $A_1$ l'insieme dei punti in $A$ che sono connessi a $\vec p_1$ per archi (anche poligonali).
	Esso non può essere già vuoto, dato che sicuramente contiene almeno un punto, che è $\vec p_1$ stesso.
	Chiamiamo $\mathscr P(\vec q_1,\vec q_2)$ un generico arco che connette i due punti $\vec q_1$ e $\vec q_2$: allora possiamo scrivere $A_1=\{\vec x\in A\colon\exists\mathscr P(\vec x,\vec p_1)\}$.
	Consideriamo un intorno $B(\vec x,\delta)$ e un punto $\vec y$ in esso: il segmento $[\vec x,\vec y]$ che lo congiunge a $\vec x$ è quindi tutto contenuto in tale intorno.
	Tale segmento può essere visto chiaramente come un arco $\mathscr P(\vec x,\vec y)$ che li connette, quindi collegando $\vec x$ a $\vec y$ e $\vec x$ a $\vec p_1$ avremo un arco che connette $\vec p_1$ a $\vec y$; in breve, $\mathscr P(\vec p_1,\vec x)\cup[\vec x,\vec y]=\mathscr P(\vec p_1,\vec y)$.
	Poich\'e questo ragionamento vale per ogni scelta di $\vec y\in B(\vec x,\delta)$, questo intorno è tutto contenuto in $A_1$, che quindi risulta aperto.

	Definiamo ora $A_2\defeq A\setminus A_1$.
	Sia $A_2\neq\emptyset$ (altrimenti il teorema è subito dimostrato), e $\vec x\in A_2$: L'insieme $A$ è aperto, quindi esiste un intorno $B(\vec x,\delta)$ che è contenuto in $A$: sia $\vec y$ un punto di questo intorno.
	Come prima, $[\vec x,\vec y]\subset B(\vec y,\delta)$.
	Supponiamo che $\vec y$ sia in $A_1$: risulterebbe allora che esiste un arco che lo connette a $\vec p_1$, e di conseguenza un arco $\mathscr P(\vec p_1,\vec y)\cup[\vec x,\vec y]$ che connette $\vec x$ a $\vec p_1$.
	Però $\vec x\in A_2$, quindi siamo giunti ad una contraddizione poiché $\vec x$ appartiene sia ad $A_1$ che $A_2$ ma per come li abbiamo definiti essi non possono avere punti in comune.
	Dobbiamo quindi eliminare l'ipotesi fatta che $\vec y\in A_1$, ma allora $\vec y\in A_2$, e come prima $\exists B(\vec y,\delta)$ contenuto in $A_2$, e anche $A_2$ è quindi aperto.

	Alla fine abbiamo quindi che $A=A_1\cup A_2$ e $A_1\cap A_2=\emptyset$, dato che sono complementari rispetto ad $A$.
	Questi due sottoinsiemi sono aperti, e $A$ è connesso, quindi proprio per la definizione di insieme connesso $A_2$ deve essere vuoto (avevamo visto che $A_1$ non lo può essere), ma allora $A_1\equiv A$, che quindi è connesso per archi.
\end{proof}
La proprietà di connessione di un insieme ci permette di estendere una conseguenza del teorema di Lagrange, che affermava che su un intervallo le funzioni sono costanti se e solo se hanno (almeno in tutto l'intervallo) derivata nulla, anche a funzioni di più variabili.
\begin{corollario}
	Se $A\subset\R^n$ è aperto e connesso, allora una funzione $f\colon A\to\R$ differenziabile è costante se e solo se $\grad f(\vec x)=\null$ per ogni $\vec x\in A$.
\end{corollario}
\begin{proof}
	Se $f$ è costante ha tutte le derivate parziali nulle, e segue subito che $\grad f=\null$ in ogni punto di $A$.
	Partiamo ora da $\grad f=\null$. Definiamo, per $\vec a\in A$, l'insieme $A_1\defeq\{\vec x\in A\colon f(\vec x)=f(\vec a)\}$, e $A_2\defeq A\setminus A_1$.
	Prendiamo $\vec x_1\in A_1$ ed un suo intorno $U(\vec x_1)\subset A_1$ (è sempre possibile, perché $A$ è aperto).
	Per il teorema di Lagrange esiste $\theta\in[0,1]$ tale per cui
	\begin{equation*}
		f(\vec x)-f(\vec x_1)=\scalar{\grad f\big(\vec x_1+\theta(\vec x-\vec x_1)\big)}{\vec x-\vec x_1}=0
	\end{equation*}
	perché il gradiente è nullo in tutto $A$, ma allora $f(\vec x)=f(\vec x_1)$ per ogni $\vec x_1\in A_1$, quindi $A_1$ è aperto.
	Allo stesso tempo, però, è anche chiuso: infatti se $\tilde{\vec x}$ è un punto di accumulazione per $A_1$, allora esiste una successione $\{\vec x_n\}\subset A_1$ tale che $\vec x_n\to\tilde{\vec x}$, ma per la continuità di $f$ si ottiene che $f(\vec x_n)\to f(\tilde{\vec x})=f(\vec a)$ e dunque $\tilde{\vec x}\in A_1$.
	Allora anche $A_2$ è aperto, ma $A$ è connesso, $A_1$ necessariamente non è vuoto e di conseguenza $A_2=\emptyset$.
\end{proof}

\section{Forme differenziali lineari}
Prendiamo l'insieme $\{\dd x_1,\dd x_2,\dots,\dd x_n\}$ come base per lo spazio $\R^n_*$ duale di $\R^n$, definendola tramite la relazione $\dd x_i(\vec e_j)=\delta_{ij}$ per ogni $i,j\in\{1,\dots,n\}$.
Per un vettore $\vec a\in\R^n$ si ha $\dd x_i(\vec a)=a_i$, ossia $\dd x_i$ fornisce la coordinata $i$-esima del vettore.
Ogni elemento di $\R^n$ può essere scritto come combinazione lineare dei vettori della base canonica, cioè $\vec y=\sum_{i=1}^ny_i\vec e_i$: analogamente ogni funzionale lineare in $\R^n_*$ può essere espresso in termini della base appena descritta, ossia $L(\vec x)=\sum_{i=1}^nl_i\dd x_i(\vec x)$ o anche $L=\sum_{i=1}^nl_i\dd x_i$.

\begin{definizione} \label{d:forma-diff-lineare}
	Sia $A\subseteq\R^n$ aperto. Una mappa $\omega\colon A\to\R^n_*$ che porta $\vec x\in A$ nel funzionale lineare $\omega(\vec x)$ è detta \emph{forma differenziale lineare}.
\end{definizione}
Con la base duale scelta in precedenza si può esprimere questa mappa come una combinazione lineare
\begin{equation*}
	\omega(\vec x)=\sum_{i=1}^na_i(\vec x)\dd x_i)
\end{equation*}
dove per ogni $i=1,\dots,n$ la funzione $a_i\colon A\to\R$ è l'$i$-esimo coefficiente della forma differenziale.
La regolarità di questi coefficienti determina quella della forma differenziale: si dice che essa è di classe $\cont{k}(A)$ se $a_i\in\cont{k}(A)$ $\forall i\in\{1,\dots,n\}$.

\section{Integrale di forme differenziali}
\begin{definizione}
	Sia $A\subseteq\R^n$ aperto e $\omega\colon A\to\R^n_*$ una forma differenziale lineare continua, e $\vphi\colon[a,b]\to A$ una curva regolare (continua e differenziabile) a tratti.
	Si definisce
	\begin{equation} 
		\int_{\vphi}\omega=\int_a^b\sum_{i=1}^na_i\big(\vphi(t)\big)\phi_i'(t)\dd t.
		\label{eq:integrale-forma-diff-lin}
	\end{equation}
\end{definizione}
	La curva $\vphi$ è a valori in $A$, quindi anch'essa ha $n$ componenti.
\begin{definizione}
	Siano $\vphi\colon[a,b]\to\R^n$ e $\vpsi\colon[c,d]\to\R^n$ due curve regolari a tratti.
	Esse si dicono \emph{equivalenti} ed \emph{equiorientate} se esiste una riparametrizzazione di $\vphi$ che porta in $\vpsi$, ossia una mappa $\tau\colon[a,b]\to[c,d]$ suriettiva, derivabile e con derivata positiva in ogni punto di $[a,b]$ tale per cui $\vphi=\vpsi\circ\tau$.
\end{definizione}
Poiché $\tau'(t)>0$ per ogni $t\in[a,b]$ segue che $\tau$ è anche iniettiva e dunque biunivoca tra $[a,b]$ e $[c,d]$.
\begin{teorema}
	Sia $\omega$ una forma differenziale dall'insieme aperto $A\subseteq\R^n$ a $R^n_*$, di classe $\cont{1}$, e $\vphi\colon[a,b]\to A$ e $\vpsi\colon [c,d]\to A$ due curve regolari a tratti, equivalenti e con sostegno in $A$.
	Se sono equiorientate, si ha $\int_{\vphi}\omega=\int_{\vpsi}\omega$; se invece sono orientate in senso opposto, ossia (seguendo la definizione precedente) $\tau'(t)<0$ $\forall t\in[a,b]$, allora $\int_{\vphi}\omega=-\int_{\vpsi}\omega$.
\end{teorema}
\begin{proof}
	Sia $\vphi=\vpsi\circ\tau$ in modo che le due curve siano equivalenti.
	Se sono anche equiorientate, allora $\tau'(t)>0$ $\forall t\in[a,b]$ e $\tau(a)=c$ e $\tau(b)=d$.
	Dunque risulta che
	\begin{equation}
		\int_{\vphi}\omega=\int_a^b\sum_{i=1}^na_i\big(\vphi(t)\big)\phi_i'(t)\dd t
		=\int_a^b\sum_{i=1}^na_i\big(\vpsi(\tau(t))\big)\psi_i'\big(\tau(t)\big)\tau'(t)\dd t,
	\end{equation}
	calcolando la derivata della funzione composta $\vphi=\vpsi\circ\tau$.
	Chiamiamo allora $r\defeq\tau(t)$ e sostituiamo le variabili nell'integrale: otteniamo che $\tau'(t)\,\dd t=\dd r$, da cui risulta
	\begin{equation}
		\int_{\vphi}\omega=\int_{\tau(a)}^{\tau(b)}\sum_{i=1}^na_i\big(\vpsi(r)\big)\psi_i'(r)\,\dd r
		=\int_c^d\sum_{i=1}^na_i\big(\vpsi(r)\big)\psi_i'(r)\,\dd t=\int_{\vpsi}\omega.
	\end{equation}
	
	Se le due curve hanno invece verso opposto, si ha $\tau'(t)<0$ e dunque la mappa $\tau$ è decrescente, e di conseguenza\footnote{Si ricordi che $\tau\colon[a,b]\to[c,d]$ e di conseguenza deve sempre risultare, almeno, che $a<b$ e $c<d$.} $\tau(a)=d$ e $\tau(b)=c$. Basta ripetere i passaggi compiuti in precedenza, per trovarsi alla fine con gli estremi di integrazione invertiti rispetto a prima, da cui si ricava facilmente che
	\begin{equation*}
		\int_{\vphi}\omega=-\int_{\vpsi}\omega.
	\end{equation*}
\end{proof}
Ritorniamo al differenziale di una funzione scalare $F\colon A\subseteq\R^n\to\R$: esso è una quantità che porta un punto di $\R^n$, che rappresenta l'incremento della variabile in una data direzione, in uno scalare ch è l'incremento della funzione associato a tale ``spostamento''.
È dunque a tutti gli effetti un funzionale lineare, cioè un elemento di $\R^n_*$, infatti lo possiamo scrivere come
\begin{equation*}
	\dd F=\sum_{i=1}^n\drp{F}{x_i}\dd x_i.
\end{equation*}
Basta prendere la derivata $\drp{F}{x_i}$ come coefficiente $a_i$ per assimilare $\dd F$ ad una forma differenziale.
Ovviamente la regolarità diminuisce: se $F\in\cont{k}(A)$, si avrà $\dd F\in\cont{k-1}(A)$.
\begin{teorema}
	 Siano $A$ un aperto di $\R^n$, $F\colon A\to\R$ di classe $\cont{1}(A)$ e $\vphi\colon[a,b]\to\R^n$ regolare a tratti. Allora
\end{teorema}
\begin{proof}
	Dalle regole di derivazione di funzioni composte abbiamo
	\begin{equation}
		\int_{\vphi}\dd F=\int_a^b\sum_{i=1}^n\drp{F}{x_i}\big(\vphi(t)\big)\phi_i'(t)\,\dd t=\int_a^b\drv{}{t}F\big(\vphi(t)\big).
	\end{equation}
	Per il teorema fondamentale del calcolo integrale quest'ultimo è uguale a $F\big(\vphi(b)\big)-F\big(\vphi(a)\big)$.
\end{proof}
Notiamo che il risultato non cambierebbe se prendessimo un'altra curva con gli stessi estremi.

\section{Forme differenziali esatte}
Tra le forme differenziali lineari ne esistono alcune di determinata importanza, e facili da calcolare.
\begin{definizione} \label{d:forma-diff-esatta}
	Sia $A\subseteq\R^n$ aperto e $\omega\colon A\to\R^n_*$ una forma differenziale lineare continua in $A$.
	Essa si dice \emph{esatta} se esiste una mappa $F\colon A\to\R$ di classe $\cont{1}(A)$ tale per cui $\dd F=\omega$.
	In tal caso, la funzione $F$ è detta \emph{potenziale} di $\omega$.
\end{definizione}
\begin{osservazione}
	Se $F$ è un potenziale di $\omega$ su $A$, allora anche $G=F+c$, dove $c$ è una qualsiasi costante, lo è.
	D'altra parte, se $F_1$ e $F_2$ sono entrambi potenziali di una medesima forma differenziale su $A$, e $A$ è connesso, allora differiscono di una costante.%Infatti $\dd(F_1-F_2)=0$ da cui\dots
	Una forma differenziale esatta ammette infiniti potenziali, tutti uguali ameno di una costante additiva (si può dunque determinare univocamente un potenziale a meno di tale costante).
\end{osservazione}
Introduciamo ora due classi di curve: la classe delle curve regolari a tratti definite da uno stesso intervallo e con estremi dati nell'insieme $A$, ossia l'insieme $\Phi_{\vec a,\vec b}(A)\defeq\{\vphi\colon[\alpha,\beta]\to A\text{ regolari a tratti e con }\vphi(\alpha)=\vec a, \vphi(\beta)=\vec b\}$, e la classe delle curve chiuse $\Phi_c(A)\defeq\{\vphi\colon[\alpha,\beta]\to A\text{ regolare a tratti e per cui }\vphi(\alpha)=\vphi(\beta)\}$, cioè con estremi coincidenti, in $A$.
Sfruttiamo questa definizione per enunciare i seguenti teoremi.
\begin{teorema}
	Sia $A\subseteq\R^n$ aperto e connesso, $\omega\colon A\to\R^n_*$ una forma differenziale lineare continua.
	Essa è esatta se e solo se $\int_{\vphi}\omega=0$ per ogni $\vphi\in\Phi_c(A)$.
\end{teorema}
\begin{teorema}
	Sia $A\subseteq\R^n$ aperto e connesso e $\omega\colon A\to\R^n_*$ continua.
	Essa è una forma differenziale esatta se e solo se $\int_{\vphi}\omega=\int_{\vpsi}\omega$ per qualsiasi coppia $\vphi,\vpsi$ nella classe $\Phi_{\vec a,\vec b}(A)$ e per ogni scelta di $\vec a$ e $\vec b$ in $A$.
\end{teorema}
\begin{proof}
	Se $\omega$ è esatta, segue dalla definizione \ref{d:forma-diff-esatta} che esiste una funzione $f$ tale per cui $\dd f=\omega$, e allora risulta $\int_{\vphi}\dd f=f\big(\vphi(\beta)\big)-f(\big(\vphi(\alpha)\big)$.
	Una curva $\vpsi\in\Phi_{\vec a,\vec b}(A)$ ha gli stessi estremi, siano essi $\vpsi(\gamma)=\vphi(\alpha)$ e $\vpsi(\delta)=\vphi(\beta)$, e dunque
	\begin{equation}
		\int_{\vpsi}\dd f=f\big(\vpsi(\delta)\big)-f\big(\vpsi(\gamma)\big)=\int_{\vphi}\dd f.
	\end{equation}
	
	Sia ora $\vec p\in A$. Poiché l'insieme è connesso per archi, per ciascun punto $\vec x\in A$ esiste una curva $\vphi$ regolare a tratti che connette i due punti, vale a dire $\vphi\in\Phi_{\vec p,\vec x}(A)$.
	Sia allora $F(\vec x)=\Int_{\vphi}\omega$: per ipotesi, l'integrale dipende soltanto dagli estremi della curva, quindi fissato $\vec p$ la $F$ è ben definita. $A$ è aperto, quindi esiste sempre un intorno $B(\vec x,\delta)\subset A$ per ogni $\vec x\in A$.
	Prendiamo un versore $\vec v\in\R^n$, e per $h\in\R$ valutiamo l'incremento $\vec q=\vec x+h\vec v$.
	Per $\norm{h}$ sufficientemente piccolo, anche $\vec q$ è in $A$. Chiamiamo allora $\vpsi$ la curva
	\begin{equation*}
		\vpsi(t)=
		\begin{cases}
			\phi(t)	&t\in[\alpha,\beta]\\
			\vec x+(t-\beta)h\vec v	&t\in(\beta,\beta+1]
		\end{cases}
	\end{equation*}
	Essa va da $\vpsi(\alpha)=\vec p$ a $\vpsi(\beta+1)=\vec x+(\beta+1-\beta)h\vec v=\vec q$, passando per $\vpsi(\beta)=\vec x$, ed è regolare a tratti con sostegno in $A$.
	Calcoliamo dunque il rapporto incrementale di $F$, ossia $F(\vec x+h\vec v)-F(\vec x)=\Int_{\vpsi}\omega-\Int_{\vphi}\omega$, che per definizione è
	\begin{equation}
		\Int_\alpha^{\beta+1}\sum_{i=1}^na_i\big(\vpsi(t)\big)\psi_i'(t)\,\dd t-\Int_\alpha^\beta\sum_{i=1}^na_i\big(\vphi(t)\big)\phi_i'(t)\,\dd t.
	\end{equation}
	Nell'intervallo $[\alpha,\beta]$, però, le due curve $\vphi$ e $\vpsi$ coincidono, quindi $a_i\big(\vphi(t)\big)=a_i\big(\vpsi(t)\big)$ per ogni $t$ in tale intervallo e per ogni $i=1,\dots,n$.
	Allora il rapporto incrementale di $F$ diventa
	\begin{equation}
		\begin{aligned}
			&\int_\alpha^\beta\sum_{i=1}^na_i\big(\vphi(t)\big)\phi_i'(t)\,\dd t+\Int_\beta^{\beta+1}\sum_{i=1}^na_i\big(\vpsi(t)\big)\psi_i'(t)\,\dd t-\Int_\alpha^\beta\sum_{i=1}^na_i\big(\vphi(t)\big)\phi_i'(t)\,\dd t=\\
			=&\Int_\beta^{\beta+1}\sum_{i=1}^na_i\big(\vpsi(t)\big)\psi_i'(t)\,\dd t=\\
			=&\Int_\beta^{\beta+1}\sum_{i=1}^na_i\big(\vec x+(t-\beta)h\vec v)hv_i\,\dd t.
		\end{aligned}
	\end{equation}
	Effettuiamo ora un cambio di variabile, ponendo $r=h(t-\beta)$, per cui si avrà $r(\beta)=0$, $r(\beta+1)=h$ e $\dd r=h\,\dd t$: l'integrale precedente diventa
	\begin{equation}
		\int_0^h\sum_{i=1}^na_i\big(\vec x+r\vec v)v_i\,\dd r.
	\end{equation}
	Dividendo per $h$ e passando al limite per $h\to0$, anche $r$ tende a zero, e dal teorema fondamentale del calcolo integrale otteniamo la derivata di $F$ in $\vec x$ lungo la direzione indicata da $\vec v$:
	\begin{equation}
		\lim_{h\to0}\frac{F(\vec x+h\vec v)-F(\vec x)}{h}=\lim_{h\to0}\frac1{h}\int_0^h\sum_{i=1}^na_i(\vec x+r\vec v)hv_i\,\dd r=\lim_{r\to 0}\sum_{i=1}^na_i(\vec x+r\vec v)v_i=\sum_{i=1}^na_i(\vec x)v_i,
	\end{equation}
	da cui si nota facilmente che, poiché la derivata direzionale si può ottenere come prodotto scalare tra il gradiente di $F$ e $\vec v$, che proprio $\grad F=(a_1,\dots,a_n)$. Segue allora che, se prendiamo la base duale introdotta all'inizio del capitolo, si ottiene
	\begin{equation}
		\dd F=\scalar{\grad F}{\dd \vec x}=\sum_{i=1}^na_i\dd x_i
	\end{equation}
	e che quindi la forma differenziale $\omega$ è il differenziale di una funzione scalare, cioè è esatta.
\end{proof}

\section{Forme differenziali chiuse}
Se una forma differenziale è esatta in un insieme $A$ connesso e aperto, ed essa è continua, allora il suo potenziale è una funzione $F\in\cont{2}(A)$: per essa vale il teorema di Schwarz, per cui le derivate miste coincidono.
Poiché la forma è esatta, $\drp{F}{x_i}(\vec x)=a_i(\vec x)$, perciò
\begin{equation}
	\drp{a_i}{x_j}(\vec x)=\frac{\partial^2 F}{\partial x_j\partial x_i}(\vec x)=\frac{\partial^2 F}{\partial x_i\partial x_j}(\vec x)=\drp{a_j}{x_i}(\vec x).
\end{equation}
Questa proprietà che le derivate miste ``incrociate'' sono uguali si ritrova nella seguente definizione.
\begin{definizione} \label{d:forma-diff-chiusa}
	Una forma differenziale lineare $\omega\in\cont{1}(A)$ per un insieme $A\in\R^n$ aperto si dice \emph{chiusa} in tale insieme se
	\begin{equation}
		\drp{a_i}{x_j}(\vec x)=\drp{a_j}{x_i}(\vec x)
	\end{equation}
	per ogni $\vec x\in A$ e $i,j\in\{1,\dots,n\}$ con $i\neq j$.
\end{definizione}
Da quanto appena detto è evidente che tutte le forme differenziali esatte sono anche chiuse.
Questo però non è solo un altro modo di definirle, poiché non tutte le forme differenziali chiuse sono esatte.
Prendiamo ad esempio la forma
\begin{equation*}
	\omega(x,y)=a_1(x,y)\,\dd x+a_2(x,y)\,\dd y=\frac{y}{x^2+y^2}\,\dd x-\frac{x}{x^2+y^2}\,\dd y
\end{equation*}
in $A=\R^2\setminus\{\vec 0\}$.
Essa è $\cont{\infty}(A)$, ed è chiusa, infatti
\begin{equation}
	\drp{}{y}\frac{y}{x^2+y^2}=\frac{x^2-y^2}{(x^2+y^2)^2}=\drp{}{x}\frac{-x}{x^2+y^2}
\end{equation}
Notiamo inoltre che la funzione $F(x,y)=\arctan\frac{x}{y}$, a meno di costanti arbitrarie, è tale per cui $\drp{F}{x}(x,y)=a_1(x,y)$ e $\drp{F}{y}(x,y)=a_2(x,y)$ per ogni $(x,y)\in A$, quindi se esiste un potenziale non può che essere della forma $F(x,y)+c$ con $c\in\R$.
A causa della singolarità per $y=0$ dobbiamo però prima dividere $A$ nei due sottoinsiemi per $y<0$ e $y>0$, in cui avremo rispettivamente $F(x,y)+c$ e $F(x,y)+d$ (a priori non possiamo dire che sono uguali).
Determiniamo queste costanti: fissato $x_0<0$, abbiamo i limiti
\begin{equation}
	\lim_{\substack{x\to x_0\\ y\to 0^+}}F(x,y)=c-\frac{\pi}2\qeq\lim_{\substack{x\to x_0\\ y\to 0^-}}F(x,y)=d+\frac{\pi}2,
\end{equation}
che devono coincidere, quindi $c=d+\pi$ e abbiamo la funzione
\begin{equation}
	F(x,y)=
	\begin{cases}
		\arctan\frac{x}{y}+c	&y>0\\
		c-\frac{\pi}2			&y=0\\
		\arctan\frac{x}{y}+c-\pi	&y<0
	\end{cases}
\end{equation}
Però se prendiamo $x_1>0$, abbiamo i limiti
\begin{equation}
	\lim_{\substack{x\to x_1\\ y\to 0^+}}F(x,y)=\frac{\pi}2+c\qeq\lim_{\substack{x\to x_1\\ y\to 0^-}}F(x,y)=-\frac{3\pi}2+c,
\end{equation}
ma allora $F$ è discontinua in $A$, perciò non può essere il potenziale di $\omega$, che quindi non ammette potenziali e non è esatta.
Il problema da superare non risiede tanto nella funzione, ma nell'insieme in cui la stiamo studiando: è ovvio che se al posto di $A$ avessimo preso solo, ad esempio, il suo sottoinsieme in cui $x>0$, non ci sarebbero stati problemi di continuità e avremmo senza problemi trovato un potenziale per $\omega$.
Dobbiamo quindi restringere la classe degli insiemi ``buoni'' che ci permettano di dire che se una forma è chiusa è anche esatta: partiamo dagli insiemi \emph{stellati}.
\begin{definizione} \label{d:insieme-stellato}
	Un insieme $A\subseteq\R^n$ aperto si dice \emph{stellato} rispetto ad un suo punto $\vec x_0$ se per ogni $\vec x\in A$ si ha $[\vec x,\vec x_0]\subset A$, cioè il segmento che li unisce è tutto contenuto in $A$.
\end{definizione}
L'insieme dell'esempio precedente, avendo un buco nell'origine, non era stellato.
\begin{teorema} \label{t:forma-chiusa-stellato}
	Sia $A\subseteq\R^n$ un insieme aperto e stellato, e $\omega$ una forma differenziale lineare in $A$. Se $\omega$ è chiusa, allora è anche esatta.
\end{teorema}
\begin{proof}
	Sia $\vec x_0\in A$, e per ogni $\vec x\in A$ prendiamo la parametrizzazione del segmento $\vphi(t)=\vec x_0+(\vec x-\vec x_0)t$, con $t\in[0,1]$: è una curva con sostegno contenuto in $A$, poich\'e è stellato.
	Definiamo $F(\vec x)\defeq\int_{\vphi}\omega$. Allora calcolando l'integrale della forma otteniamo
	\begin{equation}
		F(\vec x)=\int_0^1\sum_{i=1}^na_i\big(\vec x_0+(\vec x-\vec x_0)t\big)(x_i-x_{0,i})\,\dd t.
	\end{equation}
	Si ha  che $\omega$ è almeno di classe $\cont{1}(A)$, in quanto chiusa, quindi per la sua continuità risulta
	\begin{equation}
		\begin{split}
			\drp{F}{x_j}(\vec x)&=\drp{}{x_j}\int_0^1\sum_{i=1}^na_i\big(\vec x_0+(\vec x-\vec x_0)t\big)(x_i-x_{0,i})\,\dd t=\\
			&=\int_0^1\sum_{i=1}^n\drp{}{x_j}\Big[a_i\big(\vec x_0+(\vec x-\vec x_0)t\big)(x_i-x_{0,i})\Big]\,\dd t=\\
			&=\int_0^1\sum_{i=1}^n\drp{a_i}{x_j}\big(\vec x_0+(\vec x-\vec x_0)t\big)(x_i-x_{0,i})t\,\dd t+\int_0^1a_j\big(\vec x_0+(\vec x-\vec x_0)t\big)\,\dd t=\\
			&=\int_0^1\sum_{i=1}^n\drp{a_j}{x_i}\big(\vec x_0+(\vec x-\vec x_0)t\big)(x_i-x_{0,i})t\,\dd t+\int_0^1a_j\big(\vec x_0+(\vec x-\vec x_0)t\big)\,\dd t=\\
			&=\int_0^1t\drv{a_j}{t}\big(\vec x_0+(\vec x-\vec x_0)t\big)\,\dd t+\int_0^1a_j\big(\vec x_0+(\vec x-\vec x_0)t\big)\,\dd t=\\
			&=\int_0^1\drv{}{t}\Big[ta_j\big(\vec x_0+(\vec x-\vec x_0)t\big)\Big]\,\dd t=\\
			&=ta_j\big(\vec x_0+(\vec x-\vec x_0)t\big)\bigg|_0^1=a_j(\vec x).
		\end{split}
	\end{equation}
	Poich\'e $\drp{F}{x_j}(\vec x)=a_j(\vec x)$, risulta $\omega=\dd F$ dunque è esatta.
\end{proof}
La dimostrazione appena eseguita indica anche un metodo per individuare, se esiste, il potenziale di una forma: si prende un punto arbitrario nell'insieme dato, e sicalcola l'integrale della forma lungo un segmento che connette tale punto ad un generico $\vec x$, sempre nell'insieme.
Ad esempio, prendiamo in $A=\{(x,y)\in\R^2\colon y>x\}$ la forma differenziale
\begin{equation*}
	\omega(x,y)=a_1(x,y)\,\dd x+a_2(x,y)\,\dd y=\frac{2xy-2x^2-1}{y-x}\,\dd x+\frac{y-x+1}{y-x}\,\dd y.
\end{equation*}
Essa è di classe $\cont{\infty}(A)$, ed è chiusa: infatti
\begin{equation}
	\drp{a_1}{y}(x,y)=\drp{}{y}\bigg(2x-\frac1{y-x}\bigg)=\frac1{(y-x)^2}=\drp{a_2}{x}(x,y).
\end{equation}
L'insieme $A$ è stellato, quindi $\omega$ è anche esatta.
Prendiamo il punto $(0,1)$ e la curva $\vphi(t)=\big(tx,1+(y-1)t\big)$ che lo connette con un generico $(x,y)\in A$, con $t\in[0,1]$. Allora
\begin{equation}
	\begin{split}
		\int_{\vphi}\omega&=\int_0^1\bigg[2tx-\frac1{1+(y-1)t-tx}\bigg]x\,\dd t+\int_0^1\bigg[1+\frac1{1+(y-1)t-tx}\bigg](y-1)\,\dd t=\\
		&=\int_0^12tx^2\,\dd t+\int_0^1(y-1)\,\dd t+\int_0^1\frac{y-x-1}{1+(y-1)t-t}\,\dd t=\\
		&=x^2\int_0^12t\,\dd t+(y-1)\int_0^1\dd t+\int_0^1\frac{y-x-1}{1+(y-x-1)t}\,\dd t=\\
		&=x^2+y-1+\log\big[1+(y-x-1)t\big]\bigg|_0^1=\\
		&=x^2+y-1+\log(y-x)
	\end{split}
\end{equation}
è il suo potenziale.

Possiamo però andare oltre gli insiemi stellati e ampliare ancora la classe degli insiemi adatti, introducendo gli insiemi \emph{semplicemente connessi}: intuitivamente si possono pensare come insiemi ``senza buchi''.
Diamone una definizione rigorosa, tramite il concetto di curve omotope.
\begin{definizione} \label{d:curva-omotopica}
	Siano $\vphi_1,\vphi_2\colon[a,b]\to A\subset\R^n$ due parametrizzazioni di curve chiuse e $\cont{2}\ab$.
	La curva $\vphi_1$ si dice \emph{omotopica} a $\vphi_2$ se esiste una mappa (detta appunto \emph{omotopia}) $\vxi\colon[0,1]\times[a,b]\to A$, di classe $\cont{2}$, tale per cui $\vxi(0,t)=\vphi_1(t)$ e $\vxi(1,t)=\vphi_2(t)$, e per ogni $s\in[0,1]$ si abbia $\vxi(s,a)=\vxi(s,b)$.
\end{definizione}
La mappa dell'omotopia tra le due curve si può vedere più facilmente come una ``deformazione'' continua e derivabile che trasforma la prima curva, sempre chiusa, nella seconda.
\begin{definizione} \label{d:insieme-semplicemente-connesso}
	Un insieme $A\subseteq\R^n$ si dice \emph{semplicemente connesso} se ogni curva $\vphi\colon[a,b]\to A$, di classe $\cont{2}(A)$ e chiusa è omotopica ad un punto in $A$, ossia ad una curva degenere costante $\vpsi\colon[a,b]\to A$ per cui $\vpsi(t)=\vec c$, con $\vec c\in A$.
\end{definizione}
Con queste definizioni possiamo quindi enunciare il seguente teorema, che non dimostriamo.
\begin{teorema}[Lemma di Poincar\'e] \label{t:poincare}
	Sia $A\subseteq\R^n$ un insieme aperto e semplicemente connesso, e $\omega$ una forma differenziale lineare di classe $\cont{1}(A)$. Se $\omega$ è chiusa allora è anche esatta.
\end{teorema}
Questo risultato è molto importante, perch\'e fornisce le condizioni meno restrittive per affermare l'esattezza della forma.
% Magari qualche esempio?
