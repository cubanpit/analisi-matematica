\chapter{Serie di funzioni}
Analogamente alle serie numeriche, si può costruire una successione di somme parziali a partire da una successione di funzioni. Sia $\{f_n(x)\}$, con $n\in\N_0$ o un opportuno sottoinsieme, una successione di funzioni definite tutte su un medesimo insieme $E$; si consideri la successione $S_k(x)$ come la somma dei primi $k$ termini di $\{f_n(x)\}$
\[
S_n(x)=\sum_{n=1}^kf_n(x),
\]
ovviamente sempre definita in $E$: essa è la successione delle somme parziali.
La serie di funzioni
\[
S(x)=\ser{n}f_n(x)
\]
converge in $x_0$ se converge la serie numerica $\ser{n}f_n(x_0)$. La funzione $S(x)$, definita ovunque la serie di funzioni converga, è anche chiamata \emph{funzione somma}.

\section{Convergenza puntuale e uniforme}
Unendo la teoria delle serie numeriche con quella delle successioni di funzioni si ottengono le definizioni seguenti.
\begin{definizione}
La serie $\ser{n}f_n(x)$ converge \emph{puntualmente} in $J\subseteq E$ se $\ser{n}f_n(x_0)$ converge in ogni punto $x_0\in J$.
\end{definizione}
\begin{definizione}
La serie $\ser{n}f_n(x)$ converge \emph{assolutamente} in $J\subseteq E$ se $\ser{n}\abs{f_n(x)}$ converge $\forall x\in J$.
\end{definizione}
\begin{definizione}
La serie $\ser{n}f_n(x)$ converge \emph{uniformemente} in $J\subseteq E$ se la successione $S_k(x)$ converge uniformemente a $S(x)$ in $J$, cioè se
\[
\norm{S_k-S}_{\infty,J}\to 0\quad\text{per }n\to\pinf.
\]
\end{definizione}

Nel capitolo \ref{sec:spazi-funz} si era visto che lo spazo metrico delle funzioni limitate e quello delle funzioni continue $\cont{k}(E)$ sono spazi vettoriali, se si assume come insieme di arrivo l'intero $\R$. Da ciò segue che la funzione somma di funzioni limitate è a sua volta limitata, e analogamente è continua quando è somma di funzioni continue (dello stesso ordine $k$). Dunque la successione $S_k(x)$ delle somme parziali assume le stesse proprietà delle $f_n(x)$ di cui è somma:
\begin{itemize}
\item se $f_n$ è limitata per ogni $n$, allora $S_k$ è limitata;
\item se $f_n\in\cont{p}$ per ogni $n$, allora anche $S_k\in\cont{p}$.
\end{itemize}
Infine queste proprietà sono ereditate dalla funzione somma quando la convergenza è uniforme, come dal teorema \dots.
\begin{teorema}
Se $f_n(x)\in\cont{J}$ $\forall n$ e $S(x)=\ser{n}f_n(x)$ converge uniformemente in $J$, allora $S\in\cont{}(J)$.
\end{teorema}
\begin{teorema} \label{t:continuita_serie_conv_uniforme}
Se esiste finito il $\lim_{x\to x_0}f_n(x)=\ell_n\in\R$, per ogni $n$, e $S(x)=\ser{n}f_n(x)$ converge uniformemente in $J$, allora la serie $\ser{n}\ell_n$ converge e vale
\[
\ser{n}\ell_n=\lim_{x\to x_0}S(x)\then\ser{n}\lim_{x\to x_0}f_n(x)=\lim_{x\to x_0}\ser{n}f_n(x).
\]
\end{teorema}
\begin{teorema} \label{t:scambio_integrale_serie}
Se $f_n\in\rie\ab$ per ogni $n$ e la serie $S(x)=\ser{n}f_n(x)$ converge uniformemente in $[a,b]$, allora $S\in\rie\ab$ e vale
\[
\int_a^b\ser{n}f_n(x)\,\dd x=\ser{n}\int_a^bf_n(x)\,\dd x.
\]
\end{teorema}
\begin{teorema} \label{t:scambio_derivata_serie}
Se $f_n(x)$ è derivabile in $[a,b]$ per ogni $n$, la serie delle derivate $\ser{n}f_n'(x)$ converge uniformemente in $[a,b]$, ed esiste un punto $x_0\in[a,b]$ per cui la serie $\ser{n}f_n(x_0)$ converge, allora la serie $\ser{n}f_n(x)$ converge uniformemente in $[a,b]$ e vale
\[
S'(x)=\bigg(\ser{n}f_n(x)\bigg)'=\ser{n}f_n'(x).
\]
\end{teorema}

Poiché la successione delle somme parziali $\{S_k\}$ converge uniformemente in $J$ se e solo se soddisfa la condizione di Cauchy (come per le successioni), cioè se per ogni $\epsilon>0$ esiste in corrispondenza un $\overline{n}(\epsilon)$ tale che per ogni $n>m\geq\overline{n}$ interi si abbia
\[
\norm{S_n-S_m}_{\infty,J}=\bigg\|\sum_{k=m+1}^nf_k\bigg\|_{\infty,J}<\epsilon.
\]
Posti $m+1=p$ e $q>0$, si ottiene che la condizione è soddisfatta se $\forall p>\overline{n}(\epsilon)$ e $q>0$ interi si ha
\begin{equation} \label{eq:cc_serie_funzioni}
\bigg\|\sum_{k=p}^{p+q}f_k(x)\bigg\|_{\infty_J}<\epsilon.
\end{equation}
Da questo si ottiene una condizione necessaria per la convergenza uniforme della serie.
\begin{corollario}
La serie $\ser{n}f_n(x)$ converge uniformemente in $J$ solo se la successione $\{f_n(x)\}$ dei termini generali converge uniformemente in $J$.
\end{corollario}
\begin{proof}
Ponendo $q=0$ nella \eqref{eq:cc_serie_funzioni}, si ottiene che
\[
\bigg\|\sum_{k=p}^{p}f_k(x)\bigg\|_{\infty_J}=\norm{f_p}_{\infty_J}<\epsilon,
\]
ossia che $\forall\epsilon>0$ $\exists\overline{n}(\epsilon)$ tale per cui $\forall p>\overline{n}$ si ha $\norm{f_p-0}_{\infty,J}<\epsilon$ cioè $f_p$ (o equivalentemente $f_n$) converge uniformemente alla funzione identicamente nulla.
\end{proof}
La condizione appena data è ovviamente solamente necessaria: la serie $\ser{n}\frac1{n}$ diverge, ma il suo termine generale $f_n(x)=\frac1{n}$ converge uniformemente alla $f\equiv 0$ in tutto $\R$.

\section{Convergenza totale}
\begin{definizione}
Si dice che la serie $\ser{n}f_n(x)$ converge \emph{totalmente} in $J$ se esiste una successione numerica $\{M_n\}\subset\R$ tale che $\abs{f_n(x)}\leq M_n$ per ogni $n$ e $x\in J$, e per cui la serie $\ser{n}M_n$ converga.
\end{definizione}
La definizione continua a valere se come successione $\{M_n\}$ si prende il minimo dei maggioranti di $f_n(x)$ in $J$, che è proprio la norma uniforme: allora la convergenza é totale anche quando la serie $\ser{n}\norm{f_n}_{\infty,J}$ converge.
\begin{teorema}[di Weierstra\ss]
Se $\ser{n}f_n(x)$ converge totalmente in $J$, allora converge anche uniformemente.
\end{teorema}
\begin{proof}
Per ipotesi esiste una successione $\{M_n\}$ la cui serie converge e tale che per ogni $n$ si abbia $\norm{f_n}_{\infty,E}\leq M_n$, $\forall x\in E$. Allora la serie $\ser{n}M_n$ soddisfa la condizione di Cauchy: per ogni $\epsilon>0$ esiste $\overline{n}$ tale per cui $\forall p\geq\overline{n}$ e $q>0$ interi si abbia
\[
\sum_{n=p}^{p+q}M_n<\epsilon,
\]
ma allora per le disuguaglianze precedenti risulta
\[
\epsilon>\sum_{n=p}^{p+q}M_n\geq\sum_{n=p}^{p+q}\norm{f_n}_{\infty,E}\geq\bigg\|\sum_{n=p}^{p+q}f_n\bigg\|_{\infty,E}
\]
quindi la serie converge anche uniformemente per le \eqref{eq:cc_serie_funzioni}.
\end{proof}
La convergenza totale implica anche la convergenza assoluta, come è facile vedere. Ovviamente non valgono le implicazioni nel senso inverso: né la convergenza uniforme né quella assoluta, né entrambe (e men che meno quella puntuale) implicano la convergenza totale.
\paragraph{Esempi}
\begin{itemize}
\item La serie $\ser{n}\frac{(-1)^n}{n}$ ha come termine generale una funzione costante, e ovviamente converge puntualmente per il criterio di Leibnitz in tutto $\R$, ma non assolutamente. Non dipendendo in alcun modo dalla variabile $x$, la convergenza è sicuramente anche uniforme.
\item La serie $\ser{n}x^n$ converge puntualmente in $J=(-1,1)$, anche assolutamente essendo a termini positivi, ma non uniformemente perché $\norm{f_n}_{\infty,J}=f_n(1)=1$.
\item Considerata solo in $\R^+$, la serie di termine generale $f_n(x)=\frac1{x}\delta_{[n,n+1]}\scriptstyle(x)$ converge puntualmente ed assolutamente (la somma corrisponde in effetti all'integrale improprio $\int_1^{\pinf}\frac1{x}\,\scriptstyle\dd x$), inoltre $\norm{S-S_n}_{\infty,\R^+}\leq\frac1{n+1}\to 0$, quindi converge uniformemente. La convergenza non è però totale in quanto $\norm{f_n}_{\infty,\R^+}=\frac1{n}$ la cui serie diverge.
\end{itemize}

\section{Serie di potenze}
Le serie di potenze sono un caso particolare e importante di serie di funzioni: dati un punto $x_0\in\R$ e una successione $\{a_n\}\subset\R$, si chiama serie di potenze di centro $x_0$ e coefficiente $a_n$ la serie
\[
\serz{n}a_n(x-x_0)^n.
\]
Qualunque serie di potenze converge sempre in $x_0$, dove il termine generale è sempre nullo: dunque l'insieme di convergenza puntuale non è mai vuoto.
Tutte le serie di potenze saranno indicate in questo capitolo con $n\in\{0,1,\dots\}$, senza perdere la generalità; laddove vi siano problemi di esistenza del coefficiente $a_n$, l'indice iniziale sarà opportunamente cambiato. Ovviamente ciò non cambia il carattere della serie, ma è importante quando si intende calcolare la funzione somma.
Per comodità, tramite la traslazione $y=x-x_0$ si ottiene la serie $\serz{n}a_ny^n$, che è centrata in $y=0$ e mantiene le stesse caratteristiche qualitative della precedente, opportunamente traslate: d'ora in poi si tratteranno soltanto serie di potenze con questa scrittura.
\paragraph{Esempi}
\begin{itemize}
\item La serie $\serz{n}n^nx^n$ converge solo in $x=0$: altrove infatti $\sqrt[n]{n^nx^n}=n\abs{x}\to\pinf$ se $x\neq 0$ quindi diverge per il criterio della radice.
\item $\serz{n}\frac{x^n}{n^n}$, sempre per il criterio della radice, converge puntualmente in tutto $\R$.
\item $\serz{n}x^n$ converge puntualmente in $E=(-1,1)$, ma non uniformemente, perché $\sup_{x\in E}\abs{x^n}=f_n(1)=1^n=1$.
\end{itemize}
\begin{teorema}
Data una serie di potenze $\serz{n}a_nx^n$, se converge puntualmente in $x_0\neq 0$, allora converge assolutamente in tutto l'intervallo $(-\abs{x_0},\abs{x_0})$. La convergenza è inoltre totale, quindi anche uniforme, in ogni sottoinsieme compatto di esso.
\end{teorema}
\begin{proof}
Sia $\serz{n}a_nx^n$ convergente in $\xi\neq 0$: allora necessariamente deve risultare $\abs{a_n\xi^n}\to 0$, quindi esiste $c>0$ tale per cui per ogni $n$ si abbia $\abs{a_n\xi^n}<c$. Allora
\[
\abs{a_nx^n}=\abs{a_n\xi^n}\abs{\frac{x}{\xi}}^n\leq c\abs{\frac{x}{\xi}}^n,
\]
che è il termine generale di una serie geometrica, e converge dunque per $\abs{x/\xi}<1$ cioè per $x\in(-\abs{\xi},\abs{\xi})$. Per il teorema del confronto, $\serz{n}a_nx^n$ converge assolutamente nel medesimo intervallo.
Se poi $K$ è un sottoinsieme compatto di tale intervallo, sicuramente $\abs{x/\xi}\leq 1-\delta$ per qualche $\delta>0$, ma allora $\abs{a_nx^n}\leq c(1-\delta)^n$ quindi $\serz{n}a_nx^n$ converge totalmente.
\end{proof}
\begin{definizione}
Si chiama \emph{raggio di convergenza} di una serie di potenze $\serz{n}a_nx^n$ la quantità
\[
\rho=\sup\bigg\{x\in\R\colon\ser{n}a_nx^n\text{ è convergente}\bigg\}.
\]
\end{definizione}
Il raggio $\rho$ appartiene a $[0,\pinf]$:
\begin{itemize}
\item se $\rho=0$ la serie converge solo in 0;
\item se $\rho\notin\{0,\pinf\}$, la serie converge assolutamente in $(-\rho,\rho)$, totalmente nei sottoinsiemi compatti, non converge se $\abs{x}>\rho$;
\item se $\rho=\pinf$, la serie converge assolutamente $\forall x\in\R$ e totalmente in ogni insieme compatto.
\end{itemize}
Per determinare il raggio di convergenza si ricorre ai criteri seguenti.
Quando $\rho\neq 0$, non si può affermare nulla sul comportamento della serie al bordo di $(-\rho,\rho)$.
\begin{teorema}[Criterio di Cauchy-Hadamard]
Sia $\rho$ il raggio di convergenza della serie $\serz{n}a_nx^n$. Se
\[
\ell=\limsup_{n\to\pinf}\sqrt[n]{\abs{a_n}},
\]
allora $\rho=\frac1{\ell}$, usando l'algebra di $\Rex$.
\end{teorema}
\begin{proof}
Per il criterio \ref{t:criterio_radice_serie} della radice si ha
\[
\limsup_{n\to\pinf}\sqrt[n]{\abs{a_nx^n}}=\abs{x}\limsup_{n\to\pinf}\sqrt[n]{\abs{a_n}}\equiv \abs{x}\ell.
\]
che deve essere minore di 1, quindi $\abs{x}<1/\ell$, cioè $\rho=1/\ell$ è il raggio di convergenza.
\end{proof}
Questo criterio è sempre utilizzabile, in quanto il limite superiore di una successione, in $\Rex$, esiste sempre (teorema \ref{t:classe_limite_mai_vuota}).
\begin{teorema}[Criterio di d'Alembert]
Se $a_n\neq 0$ per ogni $n$ ed esiste il limite
\[
\lim_{n\to\pinf}\abs{\frac{a_{n+1}}{a_n}}\equiv\ell,
\]
allora il raggio di convergenza della serie $\serz{n}a_nx^n$ è $\rho=1/\ell$.
\end{teorema}
La dimostrazione è analoga a quella del teorema precedente. Questo criterio inoltre mostra che se $a_n$ ha una crescita o decrescita di tipo polinomiale il raggio di covergenza è 1.
Non è possibile stabilire a priori la convergenza ai bordi dell'intervallo di convergenza, ma il caso va studiato di volta in volta. Ad esempio:
\begin{itemize}
\item la serie $\serz{n}x^n$ converge in $(-1,1)$;
\item la serie $\serz{n}\frac{x^n}{n}$ converge in $[-1,1)$, in $x=-1$ in particolare per il criterio di Leibnitz;
\item la serie $\serz{n}\frac{x^n}{n^2}$ converge in $[-1,1]$.
\end{itemize}

\section{Proprietà delle serie di potenze}
\begin{teorema}
Una serie di potenze è continua nell'intervallo di convergenza $(-\rho,\rho)$.
\end{teorema}
\begin{proof}
In ogni punto $x_0\in(-\rho,\rho)$, si trova un intervallo compatto $[a,b]\subset(-\rho,\rho)$ con $a<x_0<b$. Essendo un insieme compatto, la serie converge totalmente in $[a,b]$, e poiché $a_nx^n$ è continua in $[a,b]$ per ogni $n$ si ha per il teorema \ref{t:continuita_serie_conv_uniforme} che anche la serie è continua in $[a,b]$, quindi in $x_0$.
La serie è quindi continua in tutti i punti $x_0\in(-\rho,\rho)$. 
\end{proof}
\begin{teorema}[di Abel]
Se la serie di potenze $\ser{n}a_nx^n$ ha un raggio di convergenza $\rho$ non nullo e converge in $x=\rho$, allora converge uniformemente nei sottoinsiemi compatti di $(-\rho,\rho]$.
\end{teorema}
La tesi vale ovviamente anhe nel caso in cui la serie converga in $x=-\rho$, per cui la convergenza è uniforme in $[-\rho,\rho)$, o anche entrambi.
Come conseguenza, la serie è anche continua sul bordo di $(-\rho,\rho)$, dove converge: ciò significa che
\[
\lim_{x\to\rho^-}\serz{n}a_nx^n=\serz{n}a_n\rho^n.
\]
È possibile calcolare l'integrale di una serie di potenze trasformandolo in una serie numerica grazie al teorema seguente.
\begin{teorema}
Sia $\serz{n}a_nx^n$ con raggio di convergenza $\rho$. Per $[\alpha,\beta]\subset(-\rho,\rho)$ vale l'uguaglianza
\[
\int_\alpha^\beta\serz{n}a_nx^n\,\dd x=\serz{n}a_n\frac{\beta^{n+1}-\alpha^{n+1}}{n+1}.
\]
\end{teorema}
\begin{proof}
L'intervallo $[\alpha,\beta]$ è un sottoinsieme compatto dell'intervallo di convergenza, in cui quindi la serie converge uniformemente, e il termine $anx^n$ è certamente integrabile secondo Riemann. È quindi possibile applicare il teorema \ref{t:scambio_integrale_serie} che permette di scambiare l'ordine di serie ed integrale. Integrando la serie termine a termine si ottiene
\[
\int_\alpha^\beta\serz{n}a_nx^n\,\dd x=\serz{n}\int_\alpha^\beta a_nx^n\,\dd x=\serz{n}a_n\int_\alpha^\beta x^n\,\dd x=\serz{n}a_n\frac{\beta^{n+1}-\alpha^{n+1}}{n+1}.
\]
\end{proof}

Dove la convergenza è uniforme, si può calcolare la derivata di una serie di potenze termine a termine: la derivata di $\serz{n}a_nx^n$ è dunque
\[
\serz{n}na_nx^{n-1}.
\]
Dato che il primo termine, corrispondente a $n=0$, è costante (vale $a_0$) la sua derivata è nulla, quindi non appare nella serie derivata; si può allora trascurare, senza alterare in alcun modo la serie, il primo termine facendo partire la serie da $n=1$, ottenendo
\[
\ser{n}na_nx^{n-1}.
\]
Scalando gli indici di 1 si riscrive quindi la serie derivata come
\[
\serz{n}(n+1)a_{n+1}x^n.
\]
Si calcola ora il suo raggio di convergenza $\rho'$, con il limite
\[
\sqrt[n]{(n+1)\abs{a_{n+1}}}=\sqrt[n]{n+1}\sqrt[n+1]{\abs{a_{n+1}}}^{\frac{n+1}{n}}.
\]
Il termine $\sqrt[n]{n+1}$ tende a 1, così come l'esponente $\frac{n+1}{n}$ nella seconda radice. Il termine $\sqrt[n+1]{\abs{a_{n+1}}}$ al limite per $n\to\pinf$ tende a $1/\rho$ dove $\rho$ è lo stesso raggio di convergenza della serie primitiva. Quindi si può formulare il teorema seguente.
\begin{teorema}
Sia $S(x)$ la funzione somma della serie $\serz{n}a_nx^n$, con raggio di convergenza non nullo: $S$ è derivabile in $(-\rho,\rho)$ e vale
\[
S'(x)=\serz{n}(n+1)a_{n+1}x^n,
\]
con lo stesso raggio di convergenza di $S(x)$.
\end{teorema}
\begin{proof}
La serie $\serz{n}a_nx^n$ ha come termine generale una funzione continua e derivabile per ogni $n$, e converge sicuramente in $x=0$. La serie delle derivate ha lo stesso raggio di convergenza, quindi converge totalmente in ogni sottoinsieme compatto di $(-\rho,\rho)$. Allora $S(x)$ può essere derivata termine a termine per il teorema \ref{t:scambio_derivata_serie} ottenendo
\[
S'(x)=\serz{n}(n+1)a_{n+1}x^n.\qedhere
\]
\end{proof}

Continuando a derivare un qualsiasi numero di volte l'intervallo di convergenza non cambia, allora al suo interno ogni serie di potenze è derivabile infinite volte: ogni serie di potenze di raggio di convergenza $\rho$ è quindi di classe $\cont{\infty}\big((-\rho,\rho)\big)$. Inoltre, posto $S(x)=\serz{n}a_nx^n$, vale
\[\begin{split}
S^{(k)}(x)	&=\serz{n}n(n-1)(n-2)\dots(n-k+1)a_nx^{n-k}=\\
			&=\sum_{n=k}^{\pinf}n(n-1)(n-2)\dots(n-k+1)a_nx^{n-k}=\\
			&=\serz{m}(m+k)(m+k-1)\dots(m+1)a_{m+k}x^m,
\end{split}\]
sostituendo $m=n-k$. Tutti i termini di indice $n<k$ sono via via eliminati dalle derivate in quanto ogni volta, il primo è costante.
In $x=0$, tutte le potenze diventano nulle ad eccezione del primo termine, che poiché $m=0$ diventa $k!a_k$, per ogni $k\geq 0$ (è l'ordine di derivazione), cioè deve risultare
\[
a_k=\frac{f^{(k)}(0)}{k!}.
\]
I coefficienti della serie sono quindi univocamente determinati dalle derivate della funzione somma, tramite l'equazione precedente.
\begin{teorema}
Data la serie di potenze $S(x)\serz{n}a_n(x-x_0)^n$ con raggio di convergenza $\rho$ non nullo, si ha che $S\in\cont{\infty}\big((x_0-\rho,x_0+\rho)\big)$ e per ogni punto $x$ in tale intervallo risulta
\begin{equation} \label{eq:serie_potenze_taylor}
S(x)=\serz{n}\frac{S^{(n)}(x_0)}{n!}(x-x_0)^n.
\end{equation}
\end{teorema}
\paragraph{Esempi}
\begin{itemize}
\item La serie $f(x)=\serz{n}x^n$ ha $\rho=1$. La sua somma è quella della serie geometrica $f(x)=\frac1{1-x}$. Integrando in $[0,x]$ si ha che
\[
\int_0^xf(t)\,\dd t=\int_0^x\frac{\dd t}{1-t}=-\log(1-x),
\]
e integrando invece termine a termine si ottiene
\[
\int_0^xf(t)\,\dd t=\int_0^x\serz{n}t^n\,\dd t=\serz{n}\int_0^xt^n\,\dd t=\serz{n}\frac{x^{n+1}}{n+1}=\ser{n}\frac{x^n}{n}.
\]
Quindi per ogni $x\in(-1,1)$ vale l'uguaglianza
\[
\ser{n}\frac{x^n}{n}=-\log(1-x).
\]
Per il criterio di Leibnitz però l'ultima serie converge anche in $x=-1$: allora è lecito valutare l'uguaglianza in questo punto, potendo così calcolare il valore della serie numerica:
\[
\ser{n}\frac{(-1)^n}{n}=-\log 2.
\]
\item La serie $g(x)=\serz{n}(-1)^nx^{2n}$ converge in $(-1,1)$. Si può riscrivere come $\serz{n}(-x^2)^n$, la cui somma è quindi $g(x)=\frac1{1+x^2}$: integrando in $[0,x]$ si ottiene
\[
\int_0^xg(t)\,\dd t=\int_0^x\frac{\dd t}{1+t^2}=\arctan x,
\]
e termine a termine
\[
\int_0^x\serz{n}(-1)^nt^{2n}\,\dd t=\serz{n}(-1)^n\int_0^xt^{2n}\,\dd t=\serz{n}(-1)^n\frac{x^{2n+1}}{2n+1}.
\]
Questa serie converge in $[-1,1]$, quindi in $x=1$ si ha
\[
\serz{n}\frac{(-1)^n}{2n+1}=\arctan 1=\frac{\pi}4
\]
che fornisce un metodo per approssimare il valore di $\pi$.
\end{itemize}

\section{Funzioni analitiche}
\begin{definizione}
Una funzione $f$ si dice \emph{analitica} in un punto $x_0$ se $f(x_0)$ coincide con lo sviluppo in serie di Taylor $T(x_0)$ in tale punto.
\end{definizione}
Ovviamente la definizione si estende anche ad un insieme, una volta che $f(x)$ coincide con $T(x)$ per ogni punto $x$ di tale insieme.
L'insieme delle funzioni analitiche in un insieme $E$ si indica convenzionalmente con $\cont{\omega}(E)$. La condizione di \emph{analiticità} è più restrittiva dell'appartenenza alla classe $\cont{\infty}$, cioè in un insieme $J$ si ha $\cont{\infty}(J)\supset\cont{\omega}(J)$.
Tutti i polinomi di grado qualunque sono infinitamente derivabili, e coincidono così come sono con il loro sviluppo in serie di Taylor, quindi sono analitici. Un controesempio è invece dato dalla funzione
\[
f(x)=\begin{dcases}e^{-\frac1{x^2}}&x\neq 0\\0&x=0\end{dcases}.
\]

\begin{teorema}
Siano $f\in\cont{\infty}\big((a,b)\big)$ e $x_0\in(a,b)$. Se esistono $\delta>0$, $c\geq 0$ e $M\geq 0$ tali che per $x\in(x_0-\delta,x_0+\delta)$ e per ogni $n\geq 0$ si abbia
\[
\abs{f^{(n)}(x)}\leq cM^n,
\]
allora $f$ è analitica in $(x_0-\delta,x_0+\delta)$.
\end{teorema}
\begin{proof}
Si scriva lo sviluppo di Taylor con resto secondo Lagrange della funzione $f$ nell'intorno di $x_0$, arrestato all'ordine $n-1$:
\[
f(x)=\sum_{k=0}^{n-1}\frac{f^{(k)}(x_0)}{k!}(x-x_0)^k+\frac{f^{(n)}(\xi)}{n!}(x-x_0)^n
\]
per $\xi\in(x_0-x,x_0+x)$. Se $x\in(x_0-\delta,x_0+\delta)$ dunque la derivata $f^{(n)}(\xi)$ è maggiorata per ipotesi da $cM^n$, mentre $\abs{x-x_0}<\delta$, quindi in modulo la differenza tra la funzione e lo sviluppo è
\[
\bigg|f(x)-\sum_{k=0}^{n-1}\frac{f^{(k)}(x_0)}{k!}(x-x_0)^k\bigg|=\abs{\frac{f^{(n)}(\xi)}{n!}(x-x_0)^n}\leq\frac{cM^n}{n!}\delta^n=c\frac{(\delta M)^n}{n!}
\]
che tende a 0 per $n\to\pinf$, quindi la funzione coincide con il suo sviluppo in serie di Taylor.
\end{proof}
\paragraph{Esempi}
\begin{itemize}
\item La funzione $f(x)=\sin x$ è di classe $\cont{\infty}(\R)$. Posto $x_0=0$, si ha che $f^{(n)}(0)=0$ se $n$ è pari, o $\pm 1$ se $n$ è dispari. Poiché quindi $\abs{f^{(n)}(0)}\leq 1$ per ogni $n\geq 0$, $f$ è analitica in un intorno di $x=0$: chiaramente il risultato si estende anche a tutto $\R$ (figura \ref{fig:taylor_sin}).
Allora per ogni $x\in\R$ si ha
\[
\sin x=\sum_{n=0}^{\pinf}(-1)^n\frac{x^{2n+1}}{(2n+1)!}.
\]
\begin{figure}
	\tikzsetnextfilename{taylor-sin}
	\centering
	\begin{tikzpicture}
		\begin{axis}[
			enlargelimits,axis on top,axis lines=middle,
			legend pos=outer north east,
			xlabel=$x$,xmin=-7,xmax=7,ymin=-2,ymax=2,
			ytick=\empty,xtick=\empty
		]
			\addplot [samples=500,color=black!15!white,domain=-2*pi:2*pi] function {x};
			\addplot [samples=500,color=black!30!white,domain=-2*pi:2*pi] function {x-(x^3)/6};
			\addplot [samples=500,color=black!45!white,domain=-2*pi:2*pi] function {x-(x^3)/6+(x^5)/120};
			\addplot [samples=500,color=black!60!white,domain=-2*pi:2*pi] function {x-(x^3)/6+(x^5)/120-(x^7)/5040};
			\addplot [samples=500,color=black!75!white,domain=-2*pi:2*pi] function {x-(x^3)/6+(x^5)/120-(x^7)/5040+(x^9)/362880};
			\addplot [samples=500,color=black!90!white,domain=-2*pi:2*pi] function {x-(x^3)/6+(x^5)/120-(x^7)/5040+(x^9)/362880-(x^11)/39916800};
			\addplot [very thick,samples=500,color=black,domain=-2*pi:2*pi] function {sin(x)};
			\legend{$n=0$,$n=1$,$n=2$,$n=3$,$n=4$,$n=5$,$\sin x$}
		\end{axis}
	\end{tikzpicture}
	\caption{Approssimazioni crescenti dello sviluppo di Taylor per $\sin x$. Si nota che lo sviluppo arrestato a $n=4$, che corrisponde alla nona potenza di $x$, già è un'ottima approssimazione in $(-\pi,\pi)$.}
	\label{fig:taylor_sin}
\end{figure}

Derivando la serie (la convergenza è necessariamente uniforme in tutto $\R$) si ottiene quindi l'uguaglianza
\[
\cos x=\sum_{n=0}^{\pinf}(-1)^n\frac{x^{2n}}{(2n!)}.
\]
\item Si consideri la funzione $f(x)=e^x$ in $x_0=0$. Ogni sua derivata vale $f^{(n)}(0)=1$, per ogni $n$, inoltre per $\abs{x}<R$ risulta, sempre per ogni $n$, $\abs{f^{(n)}(x)}\leq e^R$, essendo una funzione crescente. Dunque $e^x$ è analitica, cioè per ogni $x\in(-R,R)$ si ha
\[
e^x=\sum_{n=0}^{\pinf}\frac{x^n}{n!}.
\]
Poiché $R$ è del tutto arbitrario, la $f(x)=e^x$ è in realtà analitica in qualunque punto dell'asse reale (figura \ref{fig:taylor_exp}).
Questa uguaglianza è sfruttata per definire la funzione esponenziale anche in altri ambiti, come per i numeri complessi o per le matrici.
\end{itemize}
\begin{figure}
	\tikzsetnextfilename{taylor-exp}
	\centering
	\begin{tikzpicture}
		\begin{axis}[standard,xlabel=$x$,xmin=-5,xmax=3,ymin=-2,ymax=4,xtick=\empty,ytick=\empty]
			\addplot [samples=500,color=black!15!white,domain=-5:3] function {1};
			\addplot [samples=500,color=black!30!white,domain=-5:3] function {1+x};
			\addplot [samples=500,color=black!45!white,domain=-5:3] function {1+x+(x^2)/2};
			\addplot [samples=500,color=black!60!white,domain=-5:3] function {1+x+(x^2)/2+(x^3)/6};
			\addplot [samples=500,color=black!75!white,domain=-5:3] function {1+x+(x^2)/2+(x^3)/6+(x^4)/24};
			\addplot [samples=500,color=black!90!white,domain=-5:3] function {1+x+(x^2)/2+(x^3)/6+(x^4)/24+(x^5)/120};
			\addplot [very thick,samples=500,color=black,domain=-5:3] function {exp(x)};
			\legend{$n=0$,$n=1$,$n=2$,$n=3$,$n=4$,$n=5$,$e^x$}
		\end{axis}
	\end{tikzpicture}
	\caption{Approssimazioni crescenti dello sviluppo di Taylor per la funzione $e^x$.}
	\label{fig:taylor_exp}
\end{figure}

